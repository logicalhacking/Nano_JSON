\documentclass[10pt,DIV16,a4paper,abstract=true,twoside=semi,openright]
{scrreprt}
\usepackage[USenglish]{babel}
\usepackage[numbers, sort&compress]{natbib}
\usepackage{isabelle,isabellesym}
\usepackage{booktabs}
\usepackage{paralist}
\usepackage{graphicx}
\usepackage{amssymb}
\usepackage{xspace}
\usepackage{xcolor}
\usepackage{listings}
\usepackage[]{mathtools}
\usepackage[pdfpagelabels, pageanchor=false, plainpages=false]{hyperref}

\isabellestyle{default}
\setcounter{tocdepth}{1}
\newcommand{\ie}{i.\,e.\xspace}
\newcommand{\eg}{e.\,g.\xspace}
\newcommand{\thy}{\isabellecontext}
\renewcommand{\isamarkupsection}[1]{%
  \begingroup% 
  \def\isacharunderscore{\textunderscore}%
  \section{#1 (\thy)}%
  \endgroup% 
}

\title{Nano JSON}
\subtitle{Working with JSON formatted data in Isabelle/HOL} 
\author{Achim~D.~Brucker}%
\publishers{
  Department of Computer Science, University of Exeter, Exeter, UK\texorpdfstring{\\}{, }
  \texttt{a.brucker@exeter.ac.uk}
}

\begin{document}
  \maketitle
  \begin{abstract}
    \begin{quote}
      JSON \ldots

      \bigskip
      \noindent{\textbf{Keywords:} JSON, JavaScript Object Notation} 
    \end{quote}
  \end{abstract}
\tableofcontents


\chapter{Introduction}

% https://www.ecma-international.org/publications-and-standards/standards/ecma-404/
% ECMA-404
% The JSON data interchange syntax
% 2nd edition, December 2017 


\cite{ecma:json:2017,ietf:rfc8259-json:2017}


and identical 7159 and in 2017 RFC 8259 as updates to RFC
        4627. The JSON syntax specified by this specification and by RFC 8259 are int

JSON (JavaScript Object Notation) is a lightweight data-interchange
format. It is easy for humans to read and write. It is easy for
machines to parse and generate. It is based on a subset of the
JavaScript Programming Language Standard ECMA-262 3rd Edition -
December 1999. JSON is a text format that is completely language
independent but uses conventions that are familiar to programmers of
the C-family of languages, including C, C++, C\#, Java, JavaScript,
Perl, Python, and many others. These properties make JSON an ideal
data-interchange language.

JSON is built on two structures:
\begin{itemize}
\item A collection of name/value pairs. In various languages, this is
  realized as an object, record, struct, dictionary, hash table, keyed
  list, or associative array.
\item An ordered list of values. In most languages, this is realized
  as an array, vector, list, or sequence.
\end{itemize}

These are universal data structures. Virtually all modern programming
languages support them in one form or another. It makes sense that a
data format that is interchangeable with programming languages also be
based on these structures.


% generated text of all theories
\input{session}

{\small
  \bibliographystyle{abbrvnat}
  \bibliography{root}
}
\end{document}

%%% Local Variables:
%%% mode: latex
%%% TeX-master: t
%%% End:
